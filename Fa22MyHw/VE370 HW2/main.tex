\documentclass[11pt,a4paper]{article}

\usepackage[version=3]{mhchem} % Package for chemical equation typesetting
\usepackage{siunitx} % Provides the \SI{}{} and \si{} command for typesetting SI units
\usepackage{graphicx} % Required for the inclusion of images
\usepackage{natbib} % Required to change bibliography style to APA
\usepackage{amsmath} % Required for some math elements 
\usepackage{geometry}
\usepackage{enumerate}
\usepackage{float}
\usepackage{subfigure}
\usepackage{pdfpages}
\usepackage{siunitx}
\usepackage{fancyhdr}
\usepackage{textcomp}
\usepackage{gensymb}
\usepackage{longtable}
\usepackage{amsfonts}
\usepackage{amssymb}
\usepackage{tikz}
\renewcommand{\labelenumi}{\alph{enumi}.}
\geometry{left=2cm,right=2cm,top=3cm,bottom=3cm}

\begin{document}

\noindent\framebox[\linewidth]{\shortstack[c]{
\Large{\textbf{Homework 2}}\vspace{1mm}\\
VE370 - Intro to Computer Organization Summer 2022}}

\begin{center}

\footnotesize{\color{blue}$*$ Name: Huang Yucheng ID: 519021910885}
\end{center}

\section*{Exercise 1}
Following memory location has address \verb|0x0F000000| and content \verb|0x15C78933|
$$
\begin{array}{|c|c|c|c|}
\hline 0 & 1 & 2 & \multicolumn{1}{c|}{3} \\
\hline 33 & 89 & \mathrm{C} 7 & 15 \\
\hline
\end{array}
$$
Write RISC-V assembly instructions to load the byte \verb|C7| as a signed number into register \verb|x20|, then show the content of \verb|x20| after the operations.


\vspace{0.3cm}

\textbf{Answer:}

\verb|lui x20, 0x0F000|

\verb|lb x20, 2(x20)|

and the content of \verb|x20| after the operations should be \verb|0xFFFFFFC7|

\section*{Exercise 2}
The RISC-V assembly program below computes the factorial of a given input $\mathrm{n}$ (n!). The integer input is passed through register \verb|x12|, and the result is returned in register \verb|x10|. In the assembly code below, there are a few errors. Correct the errors.

$$
\begin{aligned}
\verb|FACT:| & \verb|addi sp, sp, 8| \\
&\verb|sw x1, 4(sp)| \\
&\verb|sw x12, 0(sp)| \\
&\verb|add x18, x0, x12| \\
&\verb|addi x5, x0, 2| \\
&\verb|bge x12, x5, L1| \\
&\verb|mul x10, x18, x10| \\
&\verb|addi sp, sp, -8| \\
&\verb|jalr x0, 0(x1)| \\
\verb|L1:| & \verb|addi x12, x12, -1| \\
&\verb|jal x1, FACT| \\
&\verb|addi x10, x0, 1| \\
&\verb|lw x12, 4(sp)| \\
&\verb|lw x1, 0(sp)| \\
&\verb|addi sp, sp, -8| \\
&\verb|jalr x0, 0(x1)| \\
\end{aligned}
$$


\vspace{0.3cm}

\textbf{Answer:}

$$
\begin{aligned}
\verb|FACT:| & {\color{red}\verb|addi sp, sp, -8|} \\
&\verb|sw x1, 4(sp)| \\
&\verb|sw x12, 0(sp)| \\
&\verb|add x18, x0, x12| \\
&{\color{red}\verb|addi x5, x0, 1|} \\
&\verb|bge x12, x5, L1| \\
&\verb|mul x10, x18, x10| \\
&{\color{red}\verb|addi sp, sp, 8|} \\
&\verb|jalr x0, 0(x1)| \\
\verb|L1:| & \verb|addi x12, x12, -1| \\
&\verb|jal x1, FACT| \\
&\verb|addi x10, x0, 1| \\
&{\color{red}\verb|lw x12, 0(sp)|} \\
&{\color{red}\verb|lw x1, 4(sp)|} \\
&{\color{red}\verb|addi sp, sp, 8|} \\
&{\color{red}\verb|mul x10, x12, x10|} \\
&\verb|jalr x0, 0(x1)| \\
\end{aligned}
$$

\section*{Exercise 3}
Consider a proposed new instruction named \verb|rpt|. This instruction combines a loop's condition check and counter decrement into a single instruction. For example,
$$
\verb|rpt x29, loop|
$$
would do the following:

$$
\begin{aligned}
\verb|if | & \verb|(x29 > 0) {|\\
&\verb|x29=x29-1;|\\
&\verb|goto loop;| \\
\verb|}|& \\
\end{aligned}
$$

1) If this instruction were to be added to the RISC-V instruction set, what is the most appropriate instruction format?

2) What is the shortest sequence of RISC-V instructions that performs the same operation?



\vspace{1cm}

\textbf{Answer:}

1) I think the J format is the most suitable because it is like factorial, using recursive function to keep calling itself.

2) 
$$
\begin{aligned}
\verb|loop: | & \verb|addi x29, x29, -1|\\
&\verb|bgt x29, x0, loop |\\
&\verb|beq x29, x0, exit |\\
\verb|exit: | & \verb|...|\\
\end{aligned}
$$



\section*{Exercise 4}
Implement the following $\mathrm{C}$ code in RISC-V assembly. Hint: Remember that the stack pointer must remain aligned on a multiple of $16 .$

$$
\begin{aligned}
\verb|int | & \verb|fib(int n){|\\
&\verb|if (n==0) return 0;|\\
&\verb|else if (n==1) return 1;| \\
&\verb|else return fib(n-1) + fib(n-2);| \\
\verb|}|& \\
\end{aligned}
$$


\vspace{0.3cm}

\textbf{Answer:}



$$
\begin{aligned}
\verb|fib: | & \verb|beq x10, x0, Exit|\\
&\verb|addi x13, x0, 1|\\
&\verb|beq x10, x13, Exit|\\
&\verb|addi sp, sp, -8|\\
&\verb|sw x1, 0(sp)|\\
&\verb|sw x10, 4(sp)|\\
&\verb|addi x10, x10, -1|\\
&\verb|jal x1, fib|\\
&\verb|lw x11, 4(sp)|\\
&\verb|sw x10, 4(sp)|\\
&\verb|addi x10, x11, -2|\\
&\verb|jal x1, fib |\\
&\verb|lw x12, 4(sp) |\\
&\verb|add x10, x10, x12 |\\
&\verb|lw x1, 0(sp) |\\
&\verb|addi sp, sp, 8 |\\
\verb|Exit: | & \verb|jalr x0, 0(x1)|\\
\end{aligned}
$$







\section*{Exercise 5}
For each function call in above problem, show the contents of the stack after the function call is made. Assume the stack pointer is originally at address \verb|0x7ffffffc|, and follow the register conventions.

\vspace{0.3cm}

\textbf{Answer:}

\begin{enumerate}
    \item Assume that the variable n is initialized to 0 or 1. Then we do not need to use the stack, the address is always \verb|0x7ffffffc|.
    \item Assume that the variable n is initialized to 2. We need to call fib itself once and the stack will grow for 8. the address will be \verb|0x7ffffff4|. After that, it will clear the stack and finally \verb|sp| will be \verb|0x7ffffffc| again.
    \item Assume that the variable n is initialized to 3. We need to call fib itself twice and the stack will grow for 16. The deepest address will be \verb|0x7fffffec|. After that, it will clear the stack and finally \verb|sp| will be \verb|0x7ffffffc| again.
    \item Assume that the variable n is initialized to n. We need to call fib itself $n-1$ times and the stack will grow for $8(n-1)$. It will copy the process for $n-1$ times.
\end{enumerate}




\section*{Exercise 6}

Given a 32-bit RISC-V machine instruction:
$$
\verb|1111 1111 0110 1010 0001 1010 1110 0011|
$$

1) What does the assembly instruction do?

2) What type of instruction is it?


\textbf{Answer:}

1) \verb|bne x20, x22, Target| and the Target is \verb|PC-12|

2) B-type

\section*{Exercise 7}

Given RISC-V assembly instruction:
$$
\verb|lw x21, -32(sp)|
$$

1) What is the corresponding binary representation?

2) What type of instruction is it?

\vspace{0.3cm}

\textbf{Answer:}

1) \verb|1111 1110 0000 0001 0010 1010 1000 0011|

2) I-type

\section*{Exercise 8}

If the RISC-V processor is modified to have 128 registers rather than 32 registers:
\begin{enumerate}[1)]
    \item Show the bit fields of an R-type format instruction assuming opcode and funcfields are not changed.
    \item What would happen to the I-type instruction if we want to keep the total number of bits for an instruction unchanged?
    \item What is the impact on the range of addresses for a beq instruction? Assume all instructions remain 32 bits long and the size of opcode and func fields don’t change.
\end{enumerate}


\vspace{0.3cm}

\textbf{Answer:}

\begin{enumerate}[1)]
    \item 
    \begin{center}
\begin{tabular}{|c|c|c|c|c|c|}
\hline funct7 & rs2 & rs 1 & funct3 & rd & opcode \\
\hline 7 bits & 7 bits & 7 bits & 3 bits & 7 bits & 7 bits\\
\hline
\end{tabular}
    \end{center}
    
    \item 
    \begin{center}
\begin{tabular}{|c|c|c|c|c|}
\hline immediate[7:0] & rs 1 & funct3 & rd & opcode \\
\hline 8 bits & 7 bits & 3 bits & 7 bits & 7 bits\\
\hline
\end{tabular}
    \end{center}
    Since we cannot change the total bits(32). The immediate will be 4 bits shorter and can only have 8 bits.
    \item Since rs2 and rs1 are both 2 bits longer. The immediate will be 4 bits shorter(from 12 to 8). As a result, the range of addresses is shorter (from $[-2^{12}, 2^{12}-1]$ to $[-2^{8}, 2^{8}-1]$)
\end{enumerate}

\section*{Exercise 9}

Convert the following assembly code fragment into machine code, assuming the
memory location of the first instruction (LOOP) is \verb|0x1000F400|

$$
\begin{aligned}
\verb|LOOP: | & \verb|blt x0, x5, ELSE|\\
&\verb|jal x0, DONE|\\
\verb|ELSE: | & \verb|addi x5, x5, -1|\\
&\verb|addi x25, x25, 2|\\
&\verb|jal x0, LOOP|\\
\verb|DONE: | & \verb|...|\\
\end{aligned}
$$


\vspace{0.3cm}

\textbf{Answer:}

\begin{enumerate}[1)]
    \item \verb|LOOP: blt x0, x5, ELSE|. It is B-type, and the immediate is 4 =  \verb|0000 0000 0100|
    
        \begin{center}
\begin{tabular}{|c|c|c|c|c|c|c|c|}
\hline imm[11] & imm[9:4] & rs2 & rs 1 & funct3 & imm[3:0] & imm[10] & opcode \\
\hline 0 & 000000 & 00101 & 00000 & 100 & 0100 & 0 & 1100011 \\
\hline
\end{tabular}
\end{center}

Finally, the machine code should be \verb|0000 0000 0101 0000 0100 0100 0110 0011|, or  \verb|0x00504463|

    \item \verb|jal x0, DONE|. It is J-type, and the immediate is 8 =  \verb|0000 0000 0000 0000 1000|
    
\begin{center}
\begin{tabular}{|c|c|c|c|c|c|c|c|}
\hline imm[19] & imm[9:0] & imm[10] & imm[18:11] & rd & opcode \\
\hline 0 & 0000001000 & 0 & 00000000 & 00000 & 1101111 \\
\hline
\end{tabular}
\end{center}

Finally, the machine code should be \verb|0000 0001 0000 0000 0000 0000 0110 1111|, or  \verb|0x0100006F|

\item \verb|ELSE: addi x5, x5, -1|. It is I-type, and the immediate is -1 =  \verb|1111 1111 1111|
    
\begin{center}
\begin{tabular}{|c|c|c|c|c|c|c|c|}
\hline imm[11:0] & rs1 & funct3 & rd & opcode \\
\hline 111111111111 & 00101 & 000 & 00101 & 0010011 \\
\hline
\end{tabular}
\end{center}

Finally, the machine code should be \verb|1111 1111 1111 0010 1000 0010 1001 0011|, or  \verb|0xFFF28293|

\item \verb|addi x25, x25, 2|. It is I-type, and the immediate is 2 =  \verb|0000 0000 0010|
    
\begin{center}
\begin{tabular}{|c|c|c|c|c|c|c|c|}
\hline imm[11:0] & rs1 & funct3 & rd & opcode \\
\hline 000000000010 & 11001 & 000 & 11001 & 0010011 \\
\hline
\end{tabular}
\end{center}

Finally, the machine code should be \verb|0000 0000 0010 1100 1000 1100 1001 0011|, or  \verb|0x002C8C93|

\item \verb|jal x0, LOOP|. It is J-type, and the immediate is -8 =  \verb|1111 1111 1111 1111 1000|
    
\begin{center}
\begin{tabular}{|c|c|c|c|c|c|c|c|}
\hline imm[19] & imm[9:0] & imm[10] & imm[18:11] & rd & opcode \\
\hline 1 & 1111111000 & 1 & 11111111 & 00000 & 1101111 \\
\hline
\end{tabular}
\end{center}

Finally, the machine code should be \verb|1111 1111 0001 1111 1111 0000 0110 1111|, or  \verb|0xFF1FF06F|

\end{enumerate}






\end{document}